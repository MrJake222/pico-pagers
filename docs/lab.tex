\documentclass[12pt]{article}

\usepackage{amssymb} % symbol kąta
\usepackage[polish]{babel} % polskie nazwy
\usepackage[T1]{fontenc} % polskie znaki
\usepackage[margin=1.0in]{geometry} % marginesy
\usepackage[utf8]{inputenc}
\usepackage{listingsutf8} % bloki kodu
\usepackage{color} % kolory
\usepackage{indentfirst} % wcięcie w pierwszej linii paragrafu
\usepackage{graphicx} % obrazy
\usepackage{float} % dla image [H]
\usepackage{amsmath,amsthm,amssymb,mathtools} % matematyka dowód
\usepackage{changepage} % matematyka dowód
\usepackage{siunitx} % wyrównanie do kropki
\usepackage{makecell} % wyrównania nagłówków
\usepackage{hyperref} % bez obwódek wokół linków
\hypersetup{
	colorlinks,
	citecolor=black,
	filecolor=black,
	linkcolor=black,
	urlcolor=magenta
}

\usepackage{algorithm}
\usepackage{algpseudocode} % pseudokod
\newtheorem{definition}{Def}
% algorytmy po polsku
\floatname{algorithm}{Algorytm}
\floatname{required}{W}
\renewcommand{\listalgorithmname}{Spis algorytmów}
%cmds
\algnewcommand\algorithmicto{\textbf{to}}
\algnewcommand\algorithmicand{\textbf{and }}
\algnewcommand\algorithmicor{\textbf{or }}
\algnewcommand\algorithmictrue{\textbf{true}}
\algnewcommand\algorithmicfalse{\textbf{false}}
\algdef{S}[FOR]{ForTo}[3]{\algorithmicfor \  $ #1 \gets #2 $ \algorithmicto \ $ #3 $ \algorithmicdo}

% bez odstępu w itemize
\let\tempone\itemize
\let\temptwo\enditemize
\renewenvironment{itemize}{\tempone\setlength{\itemsep}{0cm}}{\temptwo}

\definecolor{lbcolor}{rgb}{0.9,0.9,0.9}  
\graphicspath{ {img/} }
% obrazek {nazwa.png}{opis}
\newcommand{\imgcustomsize}[3]{
	\begin{figure}[H]
		\centering
		\includegraphics[width=#3\textwidth]{#1}
		\caption{#2}
	\end{figure}
}
\newcommand{\img}[2]{\imgcustomsize{#1}{#2}{0.8}}

% bez opisu
\newcommand{\imgnocap}[1]{
	\begin{figure}[H]
		\centering
		\includegraphics[width=0.6\textwidth]{#1}
	\end{figure}
}

% dwa obrazki
% \imgsidebyside{1}{opis 1}{2}{opis 2}
\newcommand{\imgsidebyside}[4]{
	\begin{figure}[H]
		\centering
		\begin{minipage}{.45\textwidth}
			\centering
			\includegraphics[width=1\linewidth]{#1}
			\caption{#2}
		\end{minipage}%
		\hfill
		\begin{minipage}{.45\textwidth}
			\centering
			\includegraphics[width=1\linewidth]{#3}
			\caption{#4}
		\end{minipage}
	\end{figure}
}

% tabela z wynikami {nazwa.txt}{opis}
\newcommand{\tab}[2]{
	\begin{table}[H]
		\centering
		\caption{#2}
		\vspace{0.3cm}
		\input{#1}
	\end{table}
}

% nagłowek tabeli bold
\renewcommand\theadfont{\bfseries}

\begin{document}
	\lstset{backgroundcolor=\color{lbcolor}, language=Python, inputencoding=utf8}
	
	\title{\textit{Pico Pagers} \\ Urządzenia do powiadamiania klienta w restauracji \\
		\vspace{0.2cm}
		\large Systemy Wbudowane}
	\author{Jakub Kosmydel \\
		Norbert Morawski}
	
	\maketitle
	\tableofcontents
	\pagebreak
	
	% Dokumentacja projektu powinna być opracowana starannie i przejrzyście
	% w formie tekstu z ilustracjami (wykresy, schematy, omówione ważne
	% fragmenty kodu)
	
	\section{Wprowadzenie}
		Dummy
		
    \section{Uruchomienie}
        W celu skorzystania z systemu należy go początkowo skonfigurować. Do uruchomienia potrzebne są:
        \begin{itemize}
            \item sieć WiFi (2.4 GHz),
            \item urządzenie (np. telefon, komputer) połączone z tą siecią WiFi.
        \end{itemize}
        \subsection{Konfiguracja urządzenia}
        Konfiguracja urządzenia przebiega w następujący sposób:
        \begin{enumerate}
            \item Podłączamy \emph{Pagers Server} do zasilania.
            \item Na dowolnym urządzeniu elektronicznym wyszukujemy sieć WiFi o nazwie \mbox{\emph{pagers-server}}, łączymy się z nią podając hasło \mbox{\emph{password}}.
            \item Używając przeglądarki internetowej wchodzimy na adres \url{http://192.168.4.1}.
            \item Wyszukujemy dostępne sieci WiFi używając przycisku \emph{Initiate Scan}.
            \img{config/scan_init}{Skanowanie sieci WiFi}
            \item Wybieramy naszą sieć WiFi oraz podajemy do niej hasło. Klikamy \emph{Connect} w celu połączenia się z wybraną siecią.
        \end{enumerate}

    \pagebreak

    \section{Protokół komunikacyjny}
		Zastosowaliśmy moduły komunikacyjne 433 MHz. Skłoniły nas do tego niska cena i prostota obsługi oraz brak wymaganej komunikacji zwrotnej przez nasze urządzenia.

		\imgsidebyside{dev/tx}{Nadajnik}{dev/rx}{Odbiornik}
		\subsection{Warstwa sprzętowa}
			Obsługa nadajnika/odbiornika opiera się na podłączeniu zasilania i nadawania/odbierania poprzez jeden dostępny przewód danych. Niestety te moduły zasilane są napięciem 5V i taki standard napięć stosują na wyjściu. Pi Pico toleruje tylko 3.3V. Konieczna więc była konwersja poziomów logicznych.

			\imgsidebyside{sch/tx}{Układ nadajnika}{sch/rx}{Układ odbiornika}

			Podwójne odwracanie sygnału przez tranzystory niweluje się. Na wejściu odbiornika dostajemy nieodwrócony sygnał z nadajnika.

		\subsection{Warstwa programowa}
			Pierwszą naszą próbą było wykorzystanie wbudowanej komunikacji UART. Jednak okazało się że moduły te niezbyt dobrze przenoszą niezmieniający się sygnał (co widać poniżej).

			\img{fail1}{Wiadomość po stronie nadajnika/odbiornika}
			Po ok. 70 ms układ odbiornika zaczyna generować zakłócenia, które mogą być niepoprawnie interpretowane jako sygnały komunikacji.

			\img{fail2}{Wyodrębnione zjawisko niestabilności}
			Po dokładnej analizie przebiegów odbiornika wynikło, że stan niski może być utrzymany przez ok. 160 ms, a stan wysoki przez około 72ms. Oznacza to że sygnał musi utrzymywać minimalną częstotliwość 14Hz.

			\img{fail3_good}{Fala przenoszona bez zniekształceń}
			Przy ciągłych zmianach 10Hz okazuje się wystarczające.

			Zaistniała potrzeba implementacji protokołu który utrzymywałby stałą częstotliwość fali nośnej. Przydatny w generacji takiego przebiegu jest PWM. Sterując wypełnieniem impulsu możemy przekazywać informacje binarne.
			\imgsidebyside{pgm/proto}{Protokół oparty o PWM}{pgm/proto_conf}{Konfiguracja protokołu}{0.5}
			W obecnej wersji (konfigurowalne) zastosowaliśmy podziałkę $\frac{1}{6}$ wypełnienia PWM.
			\begin{itemize}
				\item Cisza to  $\frac{3}{6}$ wypełnienia,
				\item 0 to  $\frac{1}{6}$,
				\item 1 to  $\frac{5}{6}$.
			\end{itemize}

			Odbiornik akceptuje 0 jako maksymalnie $\frac{2}{6}$ wypełnienia, a 1 jako minimalnie $\frac{4}{6}$ wypełnienia.

			Producent określa maksymalną prędkość transmisji na 9600b/s (sugerowałoby to 9600Hz, jeżeli sygnalizowanie jest dwupoziomowe). Jednak generowane krótkie sygnały niekiedy są gubione przez nadajnik.

			\imgsidebyside{bauds/9600}{Gubienie impulsów 9600Hz}{bauds/7200_2}{Gubienie impulsów 7200Hz}
			\imgsidebyside{bauds/4800_2}{Gubienie impulsów 4800Hz}{bauds/2400_2}{2400Hz}

			Dopiero przy częstotliwości 2400Hz, wszystkie krótkie impulsy dotarły do odbiornika.

			\paragraph{Strojenie}
			Na płytce odbiornika dostępna jest cewka z możliwością dostrajania. Podjęliśmy próby jej nastawienia. Udało się osiągnąć szybkość transmisji 4800 b/s. Dla wyższych częstotliwości dostrajanie nie przyniosło efektów.
			\imgsidebyside{bauds/tuning/ph_before}{Cewka przed dostrajaniem}{bauds/tuning/4800_before}{Przebieg 4800Hz przed dostrajaniem}
			\imgsidebyside{bauds/tuning/ph_after}{Cewka po dostrojeniu}{bauds/tuning/4800_after}{Przebieg 4800Hz po dostrojeniu}

			Zostaliśmy jednak przy transmisji 2400 b/s. Jest bardziej niezawodna, a szybkość nie ma dla nas wielkiego znaczenia. Nasza ramka danych ma rozmiar 16 bajtów. Przy 2400 b/s czas transmisji 1 ramki wynosi 53ms. Jest to bardzo mało w porównaniu do tego jak często będą wysyłane takie ramki.

			\subsubsection{Implementacja}
				Wysyłanie zostało zrealizowane z wykorzystaniem sprzętowego PWM i przerwania od jego przepełnienia. Częstotliwość PWM równa jest częstotliwości sygnalizowania w transmisji. Po wywołaniu przerwania przepełnienia, poziom wypełnienia ustawiany jest w zależności od następnego bitu danych. Jeżeli takiego nie ma, nadawana jest cisza.
				\imgcustomsize{pgm/pwm_tx}{Nadawanie PWM}{0.6}

				Odbieranie natomiast wykorzystuje funkcję PWM mikrokontrolera RP2040, która umożliwia uruchomienie licznika w zależności od stanu pinu (obsługiwane są tylko piny nieparzyste). Używane jest także przerwanie na tym samym pinie, które wykrywa zbocze opadające (początek bitu). Zeruje ono licznik PWM, i czeka na kolejne zbocze opadające. Przy kolejnym zboczu wartość licznika jest interpretowana.
				\imgcustomsize{pgm/pwm_rx}{Odbieranie PWM}{0.6}
	 			Koniec ramki jest sygnalizowany przerwą w transmisji (podobnie do protokołu MODBUS). 10 znaków przerwy oznacza koniec ramki, przy czym nadajnik generuje 20 znaków przerwy.

	 \section{System plików}
	 	Do implementacji przechowywania plików (głownie statycznych plików strony WWW) został użyty system plików LittleFS. Przy użyciu funkcji dostępu do pamięci Flash, zapisuje on dane w dostępnej pamięci na płytce Pi Pico.
	 	\img{pgm/littlefs}{Funkcje dostępowe do pamięci Flash, wymagane w konfiguracji LittleFS}

	 \section{WiFi}
        Sterowanie systemem odbywa się za pomocą przeglądarki internetowej. W celu skorzystania z urządzenia należy skonfigurować połączenie z siecią WiFi.


	 \section{Serwer HTTP}
	 	Powstała własna implementacja serwera HTTP. Obsługuje on metody GET oraz POST. Interpretuje parametry URL jak i format \lstinline|application/x-www-form-urlencoded| używany w formularzach. Używa LittleFS do wysyłania statycznych plików. Nacisk został położony na wygodny interfejs do obsługi serwera.
	 	\img{pgm/http_init}{Wygodny interfejs serwera HTTP}
	 	
	 \section{Kryptografia}
	 	Aby oferować podstawowy poziom bezpieczeństwa nasze rozwiązanie szyfruje wiadomości szyfrem RSA używając 32-bitowego klucza. Przesyłana jest także zaszyfrowana suma kontrolna do sprawdzania poprawności zdekodowanych danych. Funkcjonuje to analogicznie do podpisu cyfrowego

		\begin{algorithm}[H]
			\caption{Szyfrowanie}
			\begin{algorithmic}[1]
				\State (n, d) -- klucz prywatny
				\Function{Szyfruj}{dlugosc, data, enc}
				\ForTo{i}{0}{dlugosc-1}
				\State $\lstinline|enc[i]| \equiv \lstinline|data[i]|^d$ $(mod \; n)$
				\EndFor
				\EndFunction
			\end{algorithmic}
		\end{algorithm}
		
		\begin{algorithm}[H]
			\caption{Deszyfrowanie}
			\begin{algorithmic}[1]
				\State (n, e) -- klucz publiczny
				\Function{Odszyfruj}{dlugosc, data, enc}
				\ForTo{i}{0}{dlugosc-1}
				\State $\lstinline|enc[i]| \equiv \lstinline|data[i]|^e$ $(mod \; n)$
				\EndFor
				\EndFunction
			\end{algorithmic}
		\end{algorithm}
		
	 	Algorytm dzieli wejściowe dane na 16-bitowe bloki i szyfruje je do bloków 32-bitowych. Składowe kluczy mają po 32 bity. Wiadomość wydłuża się dwukrotnie. Teraz jej długość to 32 bajty.
	 	
	 \section{Uruchomienie aplikacji}
	 	Wraz z programami dostarczony jest plik \lstinline|README.md| w którym opisane jest jak skompilować i wgrać aplikacje do systemu mikroprocesorowego.

	 
\end{document}
